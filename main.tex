\documentclass[xcolor={x11names}]{beamer}
%
% Choose how your presentation looks.
%
% For more themes, color themes and font themes, see:
% http://deic.uab.es/~iblanes/beamer_gallery/index_by_theme.html
%

\usepackage{amsmath}
\usepackage[utf8]{inputenc}
\usepackage[english]{babel}
\usepackage{graphicx}
\usepackage{tcolorbox}
\usepackage{booktabs}

\usepackage{tikz}
    \usetikzlibrary{shapes.geometric, backgrounds, calc}
    \usetikzlibrary{arrows}
    \usetikzlibrary{arrows.meta}
    \usetikzlibrary{positioning}
    \usetikzlibrary{chains}

\newcommand{\K}{\mathcal{K}}
\newcommand{\KB}{KB $\mathcal{K}$}

% Begin tikz section %%%%%%%%%%%%%%%%%%%%%%%%%%%%%%%%%%%%%%%%%%%%%%%%%%%%%%%%%%%

% Set up a few colours
\colorlet{green}{DarkOliveGreen3!50}
\colorlet{red}{IndianRed3!50}
\colorlet{blue}{DeepSkyBlue3!50}

% ------------------------------------------------------------------------------
% A macro for marking coordinates (specific to the coordinate naming
% scheme used here). Swap the following 2 definitions to deactivate
% marks.
\providecommand{\cmark}[2][]{%
  \begin{pgfonlayer}{marx}
    \node [nmark] at (c#2#1) {#2};
  \end{pgfonlayer}{marx}
  }
\providecommand{\cmark}[2][]{\relax}

% ------------------------------------------------------------------------------
% A few box styles
% <on chain> *and* <on grid> reduce the need for manual relative
% positioning of nodes
\tikzset{
  base/.style={draw, on chain, on grid, align=center, minimum height=1cm, font={\small}},
  notes/.style={node distance=13cm, align=right},
  diam/.style={base, diamond, aspect=2, text width=5cm},
  diam_small/.style={base, diamond, aspect=2, text width=4cm},
  proc/.style={base, rectangle, text width=7cm},
  proc_small/.style={base, rectangle, text width=8em},
  term/.style={proc, rounded corners, text width=8cm},
  % coord node style is used for placing corners of connecting lines
  coord/.style={coordinate, on chain, on grid, node distance=55mm and 75mm},
  coord_inner/.style={coordinate, on chain, on grid, node distance=35mm and 55mm},
  coord_inner_inner/.style={coordinate, on chain, on grid, node distance=25mm and 35mm},
  % nmark node style is used for coordinate debugging marks
  nmark/.style={draw, cyan, circle, font={\sffamily\bfseries}},
  % ----------------------------------------------------------------------------
  % Connector line styles for different parts of the diagram
  norm/.style={->, draw, blue},
  free/.style={->, draw, green},
  cong/.style={->, draw, red},
  it/.style={font={\small\itshape}}
}

% For every picture that defines or uses external nodes, you'll have to
% apply the 'remember picture' style. To avoid some typing, we'll apply
% the style to all pictures.
\tikzstyle{every picture}+=[remember picture]

% By default all math in TikZ nodes are set in inline mode. Change this to
% displaystyle so that we don't get small fractions.
\everymath{\displaystyle}

% Introduce a new counter for counting the nodes needed for circling
\newcounter{nodecount}
% Command for making a new node and naming it according to the nodecount counter
\newcommand\tabnode[1]{\addtocounter{nodecount}{1} \tikz \node (\arabic{nodecount}) {#1};}

% End tikz section %%%%%%%%%%%%%%%%%%%%%%%%%%%%%%%%%%%%%%%%%%%%%%%%%%%%%%%%%%%%%

\mode<presentation>
{
  \usetheme{default}      % or try Darmstadt, Madrid, Warsaw, ...
  \usecolortheme{default} % or try albatross, beaver, crane, ...
  \usefonttheme{default}  % or try serif, structurebold, ...
  \setbeamertemplate{navigation symbols}{}
  \setbeamertemplate{caption}[numbered]
}

\title[Mining Logical Rules]{Mining Logical Rules}
\author{Omar Gutiérrez}
\institute{}
\date{\today}

\begin{document}

\begin{frame}
  \titlepage
\end{frame}

% Uncomment these lines for an automatically generated outline.
\begin{frame}{Outline}
  \tableofcontents
\end{frame}

\section{Introduction}
\subsection{Knowledge bases}

\begin{frame}{Knowledge bases}
\begin{itemize}
  \item A knowledge base (KB) is filled with facts.
  		\begin{itemize}

        \item Every fact is represented by a \textbf{relation} between a
            \textit{subject} and \textit{object}.
        \item For example: \textit{Homer} \textbf{isHusbandOf} \textit{Marge}.
  		\end{itemize}
  \item Some popular KBs are \textbf{DBpedia}, \textbf{YAGO}, \textbf{Wikidata}, \textbf{Freebase}, etc.
\end{itemize}

\end{frame}

\begin{frame}{How they are designed?}
\begin{itemize}
  \item The Resource Description Framework (RDF) is the most popular format for the semantic KBs.
  \item Every fact in KBs is known as triple.
\end{itemize}

\vskip 0.2cm

\begin{figure}
\resizebox{7.555cm}{!}{%
    \tikzset{vertex style/.style={
    draw=#1,
    thick,
    fill=#1!70,
    text=white,
    ellipse,
    minimum width=0.03125cm,
    minimum height=0.015625cm,
    font=\tiny,
    outer sep=1pt, % the usage of this option will be clear later on
  },
}

\tikzset{
  text style/.style={
    sloped, % the text will be parallel to the connection 
    text=black,
    font=\tiny,
    above
  }
}

\begin{tikzpicture}
	\node[vertex style=red] (s) {Das Flüstern des Wassers};
    \node[vertex style=blue, below right of=s, xshift=-22mm, yshift=-12mm] (BD) {Guillermo del Toro}
 edge [<-, blue] node[text style, above]{directedBy} (s);
    \node[vertex style=blue, below left of=s, xshift=22mm, yshift=-12mm] (BD) {Sally Hawkings}
 edge [<-, blue] node[text style, above]{starredBy} (s);
\end{tikzpicture}
}
\caption{Very simple example of RDF graph}
\label{fig:rdf}
\end{figure}

\end{frame}

\begin{frame}{Open and Closed World Assumption}
\subsection{Open and Closed World Assumption}

\begin{itemize}
    \item \textbf{Open World Assumption} (OWA) is assumed in relational databases.
    \item \textbf{Closed World Assumption} (CWA) is assumed in semantic knowledge bases.
        \begin{itemize}
            \item Semantic KBs do not contain negative evidence :(
        \end{itemize}
    %\item \textbf{PARTIAL CLOSED World Assumption}.
\end{itemize}

% ToDo: Add color, symbols, bold
\begin{tabular}{ l | c | c }
    \toprule
    & OWA & CWA \\
    \midrule
    Counter-examples & No & Yes \\
    \bottomrule
\end{tabular}
\end{frame}

\subsection{Mining facts}
\begin{frame}{Mining facts}
\begin{itemize}
   \item Let’s say that we know the next facts
   \begin{itemize}
   		\item \textless Homer\textgreater\  \textbf{isHusbandOf} \textless Marge\textgreater\
        \item \textless Homer\textgreater\  \textbf{wasBornIn} \textless United States\textgreater\
        \item \ldots
        \item \textless Marge\textgreater\  \textbf{wasBornIn} \textless?\textgreater
   \end{itemize}
   \item Another example:
   \begin{itemize}
   		\item \textless Bart\textgreater\  \textbf{isSonOf} \textless Homer\textgreater
        \item \textless Lisa\textgreater\  \textbf{isDaughterOf} \textless Homer\textgreater
        \item \ldots
        \item \textless Bart\textgreater\  \textbf{isBrotherOf} \textless?\textgreater
   \end{itemize}
\end{itemize}
\end{frame}

%\begin{frame}{Language bias}
%    \begin{itemize}
%        \item Language bias are constraints whose aim is \textit{limit the size of the search space}.
%    \end{itemize}
%\end{frame}

\subsection{Measures of confidence and significance}

\begin{frame}{Standard Confidence}
\begin{equation}
    \label{eq:stand_conf}
    conf(\vec{B} \implies r(x, y)) =
             \dfrac{supp(\vec{B} \implies r(x, y))}
                   {\#(x, y): \exists z_1,\ldots,z_m : \vec{B}} %\,,
\end{equation}
\end{frame}

\begin{frame}{PCA Confidence}
\begin{equation}
    \label{eq:pca_conf}
    conf_{pca}(\vec{B} \implies r(x, y)) =
                    \dfrac{supp(\vec{B} \implies r(x, y))}
                          {\#(x, y): \exists z_1,\ldots,z_m, y' :\vec{B} \land r(x, y')} \,,
\end{equation}
\end{frame}

\subsection{Head coverage}
\begin{frame}{Head coverage}
    \begin{itemize}
       \item 
    \end{itemize}
\end{frame}

\subsection{Confidence evaluation}
\begin{frame}{Confidence evaluation}
    \begin{itemize}
       \item 
    \end{itemize}
\end{frame}


%\begin{frame}{Basic idea of the Mining Process}
%	\begin{itemize}
%        %\item Association Rule Mining under Incomplete Evidence (AMIE) is one
%        %    \textit{of several} algorithms used to mine logic rules in KB.
%        \item The idea is similar to the association rule learning.
%        % \item \textbf{Let’s take a look on it}.

%        \item Use language bias (or constraints) to reduce the size of the search space.
%            \begin{itemize}
%                \item We ignore \textit{unconnected} rules:
%                    \begin{itemize}
%                        \item \textless Homer\textgreater\  \textbf{worksOn} \textless Nuclear Plant\textgreater
%                        \item \textless Moe\textgreater\  \textbf{worksOn} \textless Moe's Pub\textgreater
%                    \end{itemize}
%                    \item We ignore \textit{reflexive} rules:
%                    \begin{itemize}
%                        \item \textless Maggie\textgreater\  \textbf{isEqualTo} \textless Maggie\textgreater
%                    \end{itemize}
%            \end{itemize}
%	\end{itemize}
%\end{frame}


\section{AMIE}

\subsection{AMIE algorithm}
\begin{frame}{AMIE algorithm}
	\begin{itemize}
   		\item Association Rule Mining under Incomplete Evidence (AMIE) is one 
            \textit{of several} algorithms used to mine logic rules in KBs.
	    \begin{itemize}
            \item It is a similar idea to the association rule learning.
        \end{itemize}
        \item \textbf{Let’s take a look on it}.
	\end{itemize}
\end{frame}

\begin{frame}{AMIE}
\begin{columns}

\begin{column}{0.5\textwidth}
	\begin{itemize}
        \item While the queue of rules $q$ is not empty\ldots
        \item We dequeue a rule $r$ from $q$.
	    \begin{itemize}
            \item If $r$ meet the criteria is added to $out$.
	    \end{itemize}
	\end{itemize}
\end{column}

\begin{column}{0.5\textwidth}
    \resizebox{0.8\textwidth}{!}{%
        \pgfdeclarelayer{marx}
\pgfsetlayers{main,marx}

\begin{tikzpicture}[%
    >=triangle 60,              % Nice arrows; your taste may be different
    start chain=going below,    % General flow is top-to-bottom
    node distance=4mm and 10mm, % Global setup of box spacing
    every join/.style={norm},   % Default linetype for connecting boxes
    ]

{\small\ttfamily\selectfont
% -------------------------------------------------
% Start by placing the nodes
% Use join to connect a node to the previous one a


\node [term] () {
    KB $\mathcal{K}$, knowledge base;\\
    $minHC$, minimum head coverage;\\
    $maxLen$, size of rules;\\
    $minConf$, threshold in the confidence\\
};
\node [proc, join] () {
    out = $\langle\rangle$ \\
    q = [$r_1(x, y)$, $r_2(x, y)$,\ldots, $r_m(x,y)$]
};
\pause
\node [diam, join, fill=gray] (is_empty) {
    $\lnot q.isEmpty()$
};
\pause
\node [proc_small, join] (p6) {
    r = q.dequeue()
};

} \end{tikzpicture}

    }
\end{column}
\end{columns}
\end{frame}

\begin{frame}{AMIE}
\begin{columns}
\begin{column}{0.5\textwidth}
	\begin{itemize}
        \item If the size of the rule $r$ is not bigger than $maxLen$:
	    \begin{itemize}
            \item We apply the rule refinement and get children rules $R(r)$.
            \item For every refined rule $r_c$ in $R(r)$.
	        \begin{itemize}
                \item If $r_c$ is not in $q$ and that head coverage value is bigger than the threshold.
                % ToDo: fix error about deep
	            %\begin{itemize}
                    \item Add $r_c$ to $q$.
	            %\end{itemize}
	        \end{itemize}
	    \end{itemize}
	\end{itemize}
\end{column}

\begin{column}{0.5\textwidth}
    \resizebox{0.8\textwidth}{!}{%
        \pgfdeclarelayer{marx}
\pgfsetlayers{main,marx}

\begin{tikzpicture}[%
    >=triangle 60,              % Nice arrows; your taste may be different
    start chain=going below,    % General flow is top-to-bottom
    node distance=4mm and 10mm, % Global setup of box spacing
    every join/.style={norm},   % Default linetype for connecting boxes
    ]

{\small\ttfamily\selectfont
% -------------------------------------------------
% Start by placing the nodes
% Use join to connect a node to the previous one a


\node [diam_small, join] (check_max_len) {
    len(r) < maxLen
};
\node [proc_small, join] (refinement) {
    R(r) = Refine(r)
};
\node [diam_small, join] (R_is_empty) {
    $r_c \in R(r)$
};

\node [diam_small, join] (check) {
    hc($r_c$) $\geq$ minHC \& $r_c$ $\notin$ q
};
\node [proc_small, join] (p11) {
    q.enqueue($r_c$)
};

% marks
\node[coord_inner_inner, right= of check] (c0) {}; %\cmark{0}
\node[coord_inner_inner, below=1cm of p11] (c1) {}; %\cmark{1}
\node[coord_inner_inner, left=of p11] (c2) {}; %\cmark{2}
\node[coord_inner_inner, left=of R_is_empty] (c3) {}; %\cmark{3}

\draw[->] (check.east) -- (c0) |- (c1) -| (c3) -- (R_is_empty);
\draw[dashed, -o] (p11.west) -- (c2);

} \end{tikzpicture}

    }
\end{column}
\end{columns}
\end{frame}

\subsection{Output rules}
\begin{frame}{Output rules}
\begin{columns}
\begin{column}{0.5\textwidth}
	\begin{itemize}
        \item After dequeue the rules
	\end{itemize}
\end{column}

\begin{column}{0.5\textwidth}
    \resizebox{0.999\textwidth}{!}{%
        % -------------------------------------------------
% Set up a new layer for the debugging marks, and make sure it is on
% top
% this is a good example: https://tex.stackexchange.com/questions/254136/how-do-i-fix-spacing-on-paths-for-a-nested-tikz-flowchart

\pgfdeclarelayer{marx}
\pgfsetlayers{main,marx}
% A macro for marking coordinates (specific to the coordinate naming
% scheme used here). Swap the following 2 definitions to deactivate
% marks.
\providecommand{\cmark}[2][]{%
  \begin{pgfonlayer}{marx}
    \node [nmark] at (c#2#1) {#2};
  \end{pgfonlayer}{marx}
  } 
\providecommand{\cmark}[2][]{\relax} 
% -------------------------------------------------

\begin{tikzpicture}[%
    >=triangle 60,              % Nice arrows; your taste may be different
    start chain=going below,    % General flow is top-to-bottom
    node distance=7mm and 55mm, % Global setup of box spacing
    every join/.style={norm},   % Default linetype for connecting boxes
    ]

{\small\ttfamily\selectfont
% ------------------------------------------------------------------------------ 
% A few box styles 
% <on chain> *and* <on grid> reduce the need for manual relative
% positioning of nodes
\tikzset{
  % ----------------------------------------------------------------------------
  % Connector line styles for different parts of the diagram
  base/.style={draw, on chain, on grid, align=center, minimum height=1cm, font={\small}},
  notes/.style={node distance=13cm, align=right},
  diam/.style={base, diamond, aspect=2, text width=5cm},
  diam_small/.style={base, diamond, aspect=2, text width=4cm},
  proc/.style={base, rectangle, text width=7cm},
  proc_small/.style={base, rectangle, text width=8em},
  term/.style={proc, rounded corners, text width=8cm},
  % coord node style is used for placing corners of connecting lines
  coord/.style={coordinate, on chain, on grid, node distance=55mm and 75mm},
  % nmark node style is used for coordinate debugging marks
  nmark/.style={draw, cyan, circle, font={\sffamily\bfseries}},
  % ----------------------------------------------------------------------------
  % Connector line styles for different parts of the diagram
  norm/.style={->, draw, blue},
  free/.style={->, draw, green},
  cong/.style={->, draw, red},
  it/.style={font={\small\itshape}}
  % ----------------------------------------------------------------------------
}
% -------------------------------------------------
% Start by placing the nodes
% Use join to connect a node to the previous one a

\node [term] (p0) {
    rule $r$;\\
    $out$, ;\\
    $minC$, threshold in the confidence\\
};
\node [diam, join, fill=blue] (is_closed) {
    $\lnot isClosed(r)$ or $conf_{pca}(r) < minC$
};
\node [proc, join, fill=red] (parents) {
    $parents$ = $parentsOfRule(r, out)$
};
\node [diam, join, fill=red] (is_empty) {
    $\lnot$ parents.isEmpty()
};
\node [proc_small, join, fill=red] (parents_dequeue) {
    $r_p$ = parents.dequeue()
};
\node [diam_small, join, fill=red] (check_confidence) {
    $conf_{pca}(r) \leq conf_{pca}(r_{p})$
};

\node [proc_small] (end) {end};
\node [proc_small, left=of end] (false) { return $false$ };
\node [proc_small, right=of end] (true) { return $true$ };

% marks
\node[coord, left=of is_closed] (c0) {}; \cmark{0} 
\node[coord, right=of is_empty] (c1) {}; \cmark{1} 
\node[coord, left=of check_confidence] (c2) {}; \cmark{2} 
\node[coord, right=of check_confidence] (c3) {}; \cmark{3} 

% labels
\path (is_closed.south) to node [midway, xshift=1em] {no} (parents); 
\path (is_closed.west) to node [midway, yshift=1em] {yes} (c0); 
\draw [*->] (is_closed.west) -- (c0) -- (c2) |- (false);

\path (is_empty.south) to node [midway, xshift=2.4em] {not empty} (parents_dequeue); 
\path (is_empty.east) to node [midway, yshift=1em] {empty} (c1); 
\draw [o->] (is_empty.east) -- (c1) -- (c3) |- (true);

\path (check_confidence.west) to node [midway, yshift=1em] {yes} (c2); 
\draw [*->] (check_confidence.west) -- (c2);

\path (check_confidence.east) to node [midway, yshift=1em] {no} (c3); 
\draw [o->] (check_confidence.east) -- (c3);

\draw[->] (false.east)  -- (end.west);
\draw[->] (true.west)  -- (end.east);

}
\end{tikzpicture}

    }
\end{column}
\end{columns}
\end{frame}

\subsection{Rule refinement}
\begin{frame}{Rule refinement}
    \begin{itemize}
        \item To explore the search space three \textit{mining operators} are used:
        \begin{enumerate}
            \item 
            \item 
            \item 
        \end{enumerate}
    \end{itemize}
\end{frame}

\subsection{In-memory database and queries}
\begin{frame}{In-memory database and queries}
\begin{figure}
    \usetikzlibrary{shapes.multipart,positioning}
\newcommand{\fourdigits}[1]{%
\ifnum #1<10 0%
\fi%
\ifnum #1<100 0%
\fi%
\ifnum #1<1000 0%
\fi% 
\number #1
}
\tikzset{bucket/.style={draw,rectangle split,rectangle split
horizontal,rectangle split parts=#1,text width=.5cm,anchor=west},
mybox/.style={rectangle,draw,minimum width=1.5cm,minimum height=0.5cm}}
\newcommand{\mybucketfour}[7][]{
\node[bucket=4,#1] (#2){#4
\nodepart{two}
#5
\nodepart{three}
#6
\nodepart{four}
#7
};
\node[draw,above left=0pt of #2.north west,anchor=south west,fill=gray!30,
text width=.5cm,yshift=-\pgflinewidth]{\textbf{#3}};
}
\newcommand{\Connect}[3][-latex]{\draw[#1] (#2) to[out=0,in=180] (#3);
}

\begin{tikzpicture}[font=\sffamily,node distance=1.5cm]
    % directory
    \begin{scope}[xshift=-4cm,yshift=2cm]   
    \node[mybox,label=left:{S R O}, alias=directory-a] (box-a) {\textbf{S}};
    \node[mybox,below=0.2cm of box-a,label=left:{S O R}, alias=directory-b] (box-b) {\textbf{S}};
    \node[mybox,below=0.2cm of box-b,label=left:{R S O}, alias=directory-c] (box-c) {\textbf{R}};
    \node[mybox,below=0.2cm of box-c,label=left:{R O S}, alias=directory-d] (box-d) {\textbf{R}};
    \node[mybox,below=0.2cm of box-d,label=left:{O S R}, alias=directory-e] (box-e) {\textbf{O}};
    \node[mybox,below=0.2cm of box-e,label=left:{O R S}, alias=directory-f] (box-f) {\textbf{O}};
    \end{scope}
    % buckets
    \mybucketfour{RO}{R}{$o_0$}{$o_1$}{$\ldots$}{$o_n$}
    \mybucketfour[below=of RO.west,anchor=west]{OR}{O}{$r_0$}{$r_1$}{$\ldots$}{$r_n$}
    \mybucketfour[below=of OR.west,anchor=west]{SO}{O}{$r_0$}{$r_1$}{$\ldots$}{$r_n$}
    % links
    \Connect{directory-a}{RO}
    \Connect{directory-b}{OR}
    \Connect{directory-c}{SO}
    %\Connect{directory-0011}{D}
    %\Connect{directory-0100}{A2}
    %\Connect{directory-0101}{B}
\end{tikzpicture}

\caption{Diagram of the database}
\label{fig:db}
\end{figure}
\end{frame}

\begin{frame}{Size queries}
\end{frame}

\begin{frame}{Count queries}
    \begin{figure}
    \resizebox{!}{0.75\textheight}{%
        % -------------------------------------------------
% Set up a new layer for the debugging marks, and make sure it is on
% top
% this is a good example: https://tex.stackexchange.com/questions/254136/how-do-i-fix-spacing-on-paths-for-a-nested-tikz-flowchart

\pgfdeclarelayer{marx}
\pgfsetlayers{main,marx}
% A macro for marking coordinates (specific to the coordinate naming
% scheme used here). Swap the following 2 definitions to deactivate
% marks.
\providecommand{\cmark}[2][]{%
  \begin{pgfonlayer}{marx}
    \node [nmark] at (c#2#1) {#2};
  \end{pgfonlayer}{marx}
  }
\providecommand{\cmark}[2][]{\relax}
% -------------------------------------------------

\begin{tikzpicture}[%
    >=triangle 60,              % Nice arrows; your taste may be different
    start chain=going below,    % General flow is top-to-bottom
    node distance=7mm and 55mm, % Global setup of box spacing
    every join/.style={norm},   % Default linetype for connecting boxes
    ]

{\small\ttfamily\selectfont
% ------------------------------------------------------------------------------
% Start by placing the nodes
% Use join to connect a node to the previous one a

\node [term] () {
    selection variable, $x$;\\
    a projection atom attached to a conjuctive query, $R(X, Y) \land B_1 \land B_2 \land \ldots \land B_n$;\\
    a threshold $k$;\\
    knowledge base, \KB;\\
};
\node [term, join] () {
    $map = \{\}$\\
    $q = B_1 \land B_2 \land \ldots \land B_n$
};
\node [diam, join] (x_in_B) {
    $x \in R(X, Y)$
};

\node[coord, left=of x_in_B] (c0) {}; \cmark{0}
\node[coord, right=of x_in_B] (c1) {}; \cmark{1}

%%%%%%%%%%%%%%%%%%%%%%%%%%%%%%%%%%%%%%%%%%%%%%%%%%%%%%%%%%%%%%%%%%%%%%%%%%%%%%%%
% JA
%%%%%%%%%%%%%%%%%%%%%%%%%%%%%%%%%%%%%%%%%%%%%%%%%%%%%%%%%%%%%%%%%%%%%%%%%%%%%%%%

\node [diam_small, below=of c0, fill=red] (yes_) {
    instantiations\\
    $r(x, y) \in R(X, Y)$
};
\node [proc, join, fill=red, label={a}] () {
    $q' = q$\\
    replace R by r, X by x, Y by y\\
};
\node [diam_small, join, fill=red] (exists) {
    \textsc{exists}(q', $\K$)
};
\node [proc_small, join, fill=red] (yes_result) {
    $map[x]++$
};

%%%%%%%%%%%%%%%%%%%%%%%%%%%%%%%%%%%%%%%%%%%%%%%%%%%%%%%%%%%%%%%%%%%%%%%%%%%%%%%%
% NEIN
%%%%%%%%%%%%%%%%%%%%%%%%%%%%%%%%%%%%%%%%%%%%%%%%%%%%%%%%%%%%%%%%%%%%%%%%%%%%%%%%

\node [diam_small, below=of c1, fill=green] (no_) {
    instantiations\\
    $r(x, y) \in R(X, Y)$
};
\node [proc, join, fill=green] () {
    $q' = q$\\
    replace R by r, X by x, Y by y\\
};
\node [proc, join, fill=green] () {
    $\mathcal{X} = \textsc{select}(x, q', \K)$
};
\node [diam_small, join, fill=green] (x_in_X) {
    $x \in \mathcal{X}$
};
\node [proc_small, join, fill=green] (no_result) {
    $map[x]++$
};

% marks
\node[coord_inner, left=of yes_result] (c3) {}; \cmark{3}
\node[coord_inner, below right=of exists] (c5) {}; \cmark{5}

\node[coord, below left=of x_in_X] (c6) {}; \cmark{6}

\node[coord_inner, right=of no_] (c8) {}; \cmark{8}

\node[coord_inner_inner, right=of x_in_X] (c2) {}; \cmark{2}
\node[coord_inner_inner, below=1cm of no_result] (c4) {}; \cmark{4}
\node[coord_inner, left=of no_result] (c9) {}; \cmark{9}

% ending nodes
\node [proc, below=1.5cm of c6] (return) {$map = \{ x \rightarrow n\} \in map: n \geq k$};
\node [proc, join] () {return $map$};
\node [proc, join] () {end};

% labels
\path (x_in_B.west) to node [midway, yshift=1em] {yes} (c0);
\draw [*->] (x_in_B.west) -- (c0) -- (yes_.north);

\path (x_in_B.east) to node [midway, yshift=1em] {no} (c1);
\draw [o->] (x_in_B.east) -- (c1) -- (no_.north);

\draw [dashed, -o] (yes_result.west) -- (c3);
\draw [->] (c3) |- (yes_.west);
\draw [-] (exists.east) -| (c5) -| (c3);

\draw [->] (yes_.east) -| (return.north);
\draw [-o] (no_.east) -| (c8) |- (c6);

\draw [dashed, -o] (no_result.west) -- (c9);
\draw [->] (x_in_X.east) -- (c2) |- (c4) -| (c9) |- (no_.west);

}
\end{tikzpicture}

    }
    \caption{Flowchart of the count projection query algorithm}
    \label{fig:count_projection}
\end{figure}
\end{frame}

\section{AMIE+}
\begin{frame}{AMIE+: Enhancements}
	\begin{itemize}
   		\item Some stop conditions were added 
	\end{itemize}
\end{frame}

\subsection{Speeding up rule refinements}
\begin{frame}{Speeding up rule refinements}

    \begin{block}{Maximum rule length}
        We do not apply a \textit{mining operator} if the size of the generated rule is 
        larger than the \textbf{$maxLen$} threshold.
	\end{block}

    % ToDo:
    % - Find an analogy for this cases, like you do not bet if you know that you are going to lose...
    % - Make an example
	\begin{itemize}
        \item If the size of the rule is \textbf{$maxLen -1$}\ldots
	        \begin{itemize}
                \item avoid the $\mathcal{O}_\mathcal{D}$ because the results is a non-closed rule.
	        \end{itemize}
        \item If the size of the rule is \textbf{$maxLen -1$} and have more than two non-closed variables\ldots
	        \begin{itemize}
                \item avoid the $\mathcal{O}_\mathcal{C}$ because we can only close at most two variables.
                \item avoid the $\mathcal{O}_\mathcal{I}$ because of the same reason.
	        \end{itemize}
	\end{itemize}

\end{frame}

\begin{frame}{Speeding up rule refinements}

    \begin{block}{Perfect rule}
        We stop adding atoms when the PCA confidence is 100\%
	\end{block}

    \begin{block}{Simplyfing projection queries}
        When a new dangling atom is added.
	\end{block}

\end{frame}

\begin{frame}{Speeding up the confidence evaluation}

    \begin{block}{}
        A \textbf{major change} in AMIE+ 
	\end{block}

    \begin{block}{Approximate the value of PCA confidence}
	\end{block}

    \begin{equation}
        \label{eq:approx_stand_conf}
        \widehat{conf}_{pca}(R) = \dfrac{supp(R)}{\widehat{d}_{pca}(R)}\,,
    \end{equation}

\end{frame}

\begin{frame}{When to use approximation?}
    % in the paper, intermediate values = existentially quantified variables
    \begin{block}{}
        Only when the head variables $x$ and $y$ are \textit{chained}
        in the body by intermediate variables:
	\end{block}
    % ToDo: Improve with tikz
    \begin{equation}
        \label{eq:candidate_for_approx}
        r_1(x, z_1) \land r_2(z_1, z_2) \land \ldots \land r_n(z_{n-1}, y) \implies r_h(x, y)\,,
    \end{equation}

    e.g.,

    \begin{equation}
        directed(x, y) \land hasActor(y, z) \implies isMarried(x, y)
    \end{equation}

\end{frame}

\begin{frame}{Compute the approximation}
\end{frame}

\section{Conclusions}

\begin{frame}{AMIE vs ILP}
    % ToDo: Add color, symbols, bold
    \begin{tabular}{ l | c | c }
        \toprule
        & AMIE & ILP \\
        \midrule
        Counter-examples & Works in the absense of them & Works with them \\
        Scalability      & Yes  & No\\
        \bottomrule
    \end{tabular}
\end{frame}

\begin{frame}{Conclusions}
%	\begin{itemize}
%   		\item Further work:
%	    \begin{itemize}
%   		    \item Make a Python implementation.
%	    \end{itemize}
%	\end{itemize}
\end{frame}

\end{document}
