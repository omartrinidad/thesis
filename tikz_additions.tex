
% Set up a few colours
\colorlet{green}{DarkOliveGreen3!50}
\colorlet{red}{IndianRed3!50}
\colorlet{blue}{DeepSkyBlue3!50}

% ------------------------------------------------------------------------------
% A macro for marking coordinates (specific to the coordinate naming
% scheme used here). Swap the following 2 definitions to deactivate
% marks.
\providecommand{\cmark}[2][]{%
  \begin{pgfonlayer}{marx}
    \node [nmark] at (c#2#1) {#2};
  \end{pgfonlayer}{marx}
  }
\providecommand{\cmark}[2][]{\relax}

% ------------------------------------------------------------------------------
% A few box styles
% <on chain> *and* <on grid> reduce the need for manual relative
% positioning of nodes
\tikzset{
  base/.style={draw, on chain, on grid, align=center, minimum height=1cm, font={\small}},
  notes/.style={node distance=13cm, align=right},
  diam/.style={base, diamond, aspect=2, text width=5cm},
  diam_small/.style={base, diamond, aspect=2, text width=4cm},
  proc/.style={base, rectangle, text width=7cm},
  proc_small/.style={base, rectangle, text width=8em},
  term/.style={proc, rounded corners, text width=8cm},
  % coord node style is used for placing corners of connecting lines
  coord/.style={coordinate, on chain, on grid, node distance=55mm and 75mm},
  coord_inner/.style={coordinate, on chain, on grid, node distance=35mm and 55mm},
  coord_inner_inner/.style={coordinate, on chain, on grid, node distance=25mm and 35mm},
  % nmark node style is used for coordinate debugging marks
  nmark/.style={draw, cyan, circle, font={\sffamily\bfseries}},
  % ----------------------------------------------------------------------------
  % Connector line styles for different parts of the diagram
  norm/.style={->, draw, blue},
  free/.style={->, draw, green},
  cong/.style={->, draw, red},
  it/.style={font={\small\itshape}}
}

% For every picture that defines or uses external nodes, you'll have to
% apply the 'remember picture' style. To avoid some typing, we'll apply
% the style to all pictures.
\tikzstyle{every picture}+=[remember picture]

% By default all math in TikZ nodes are set in inline mode. Change this to
% displaystyle so that we don't get small fractions.
\everymath{\displaystyle}

% Introduce a new counter for counting the nodes needed for circling
\newcounter{nodecount}
% Command for making a new node and naming it according to the nodecount counter
\newcommand\tabnode[1]{\addtocounter{nodecount}{1} \tikz \node (\arabic{nodecount}) {#1};}
