\documentclass{beamer}
%
% Choose how your presentation looks.
%
% For more themes, color themes and font themes, see:
% http://deic.uab.es/~iblanes/beamer_gallery/index_by_theme.html
%

\usepackage{graphicx}
\usepackage{color}
\usepackage{xcolor}
\usepackage{tikz}
    \usetikzlibrary{shapes.geometric, backgrounds, calc}

\mode<presentation>
{
  \usetheme{default}      % or try Darmstadt, Madrid, Warsaw, ...
  \usecolortheme{default} % or try albatross, beaver, crane, ...
  \usefonttheme{default}  % or try serif, structurebold, ...
  \setbeamertemplate{navigation symbols}{}
  \setbeamertemplate{caption}[numbered]
}

\usepackage[english]{babel}
\usepackage[utf8x]{inputenc}

\title[Mining Logical Rules]{Mining Logical Rules}
\author{Omar Gutiérrez}
\institute{}
\date{\today}

\begin{document}

\begin{frame}
  \titlepage
\end{frame}

% Uncomment these lines for an automatically generated outline.
%\begin{frame}{Outline}
%  \tableofcontents
%\end{frame}

\section{Introduction}

\begin{frame}{Introduction}
\begin{itemize}
  \item A knowledge base (KB) is filled with facts.
  		\begin{itemize}

        \item Every fact is represented by a \textbf{relation} between a
            \textit{subject} and \textit{object}.
        \item For example: \textit{Homer} \textbf{isHusbandOf} \textit{Marge}.
  		\end{itemize}
  \item Some popular KBs are \textbf{DBpedia}, \textbf{YAGO}, \textbf{Wikidata}, \textbf{Freebase}, etc.
\end{itemize}

\end{frame}

\begin{frame}{How they are designed?}
\begin{itemize}
  \item The Resource Description Framework (RDF) is the most popular format for the semantic KBs.
  \item Every fact in KBs is known as triple.
\end{itemize}

\vskip 0.2cm

\begin{figure}
\resizebox{7.555cm}{!}{%
    \tikzset{vertex style/.style={
    draw=#1,
    thick,
    fill=#1!70,
    text=white,
    ellipse,
    minimum width=0.03125cm,
    minimum height=0.015625cm,
    font=\tiny,
    outer sep=1pt, % the usage of this option will be clear later on
  },
}

\tikzset{
  text style/.style={
    sloped, % the text will be parallel to the connection 
    text=black,
    font=\tiny,
    above
  }
}

\begin{tikzpicture}
	\node[vertex style=red] (s) {Das Flüstern des Wassers};
    \node[vertex style=blue, below right of=s, xshift=-22mm, yshift=-12mm] (BD) {Guillermo del Toro}
 edge [<-, blue] node[text style, above]{directedBy} (s);
    \node[vertex style=blue, below left of=s, xshift=22mm, yshift=-12mm] (BD) {Sally Hawkings}
 edge [<-, blue] node[text style, above]{starredBy} (s);
\end{tikzpicture}
}
\caption{Very simple example of RDF graph}
\label{fig:rdf}
\end{figure}

\end{frame}

\begin{frame}{Open and Closed World Assumption}
\begin{itemize}
    \item \textbf{Open World Assumption} (OWA) is assumed in relational databases.
    \item \textbf{Closed World Assumption} (CWA) is assumed in semantic knowledge bases.
        \begin{itemize}
            \item Semantic KBs do not contain negative evidence :(
        \end{itemize}
    %\item \textbf{PARTIAL CLOSED World Assumption}.
\end{itemize}
\end{frame}


\section{Mining facts}
\begin{frame}{Mining facts}
\begin{itemize}
   \item Let’s say that we know the next facts
   \begin{itemize}
   		\item \textless Homer\textgreater\  \textbf{isHusbandOf} \textless Marge\textgreater\
        \item \textless Homer\textgreater\  \textbf{wasBornIn} \textless United States\textgreater\
        \item \ldots
        \item \textless Marge\textgreater\  \textbf{wasBornIn} \textless?\textgreater
   \end{itemize}
   \item Another example:
   \begin{itemize}
   		\item \textless Bart\textgreater\  \textbf{isSonOf} \textless Homer\textgreater
        \item \textless Lisa\textgreater\  \textbf{isDaughterOf} \textless Homer\textgreater
        \item \ldots
        \item \textless Bart\textgreater\  \textbf{isBrotherOf} \textless?\textgreater
   \end{itemize}
\end{itemize}
\end{frame}

\begin{frame}{Language bias}
    \begin{itemize}
        \item Language bias are constraints whose aim is \textit{limit the size of the search space}.
    \end{itemize}
\end{frame}

%\begin{frame}{Basic idea of the Mining Process}
%	\begin{itemize}
%        %\item Association Rule Mining under Incomplete Evidence (AMIE) is one
%        %    \textit{of several} algorithms used to mine logic rules in KB.
%        \item The idea is similar to the association rule learning.
%        % \item \textbf{Let’s take a look on it}.
%        \item Use language bias (or constraints) to reduce the size of the search space.
%            \begin{itemize}
%                \item We ignore \textit{unconnected} rules:
%                    \begin{itemize}
%                        \item \textless Homer\textgreater\  \textbf{worksOn} \textless Nuclear Plant\textgreater
%                        \item \textless Moe\textgreater\  \textbf{worksOn} \textless Moe's Pub\textgreater
%                    \end{itemize}
%                    \item We ignore \textit{reflexive} rules:
%                    \begin{itemize}
%                        \item \textless Maggie\textgreater\  \textbf{isEqualTo} \textless Maggie\textgreater
%                    \end{itemize}
%            \end{itemize}
%	\end{itemize}
%\end{frame}

%\section{AMIE}
%\begin{frame}{AMIE algorithm}
%	\begin{itemize}
%   		\item Association Rule Mining under Incomplete Evidence (AMIE) is one \textit{of several} algorithms used to mine logic rules in KB.
%        \item It is a similar idea to the association rule learning.
%        \item \textbf{Let’s take a look on it}.
%        \item Use language bias (or constraints) to reduce the size of the search space.
%        \begin{itemize}
%        \item
%        \end{itemize}
%	\end{itemize}
%\end{frame}

\section{Conclusions}
\begin{frame}{Conclusions}
	\begin{itemize}
   		\item Why this?
        \begin{itemize}
            \item KBs, such as, DBpedia or Wikidata are \textit{always growing}, we
                need to be sure that the incoming new facts are reliable.
            \item We can find data modeled as KBs in several areas:
            \begin{itemize}
                \item Medicine
                \item Bioinformatics
                \item Publishing
            \end{itemize}
            \item Because is interesting ;)
        \end{itemize}
	\end{itemize}
\end{frame}

\end{document}
