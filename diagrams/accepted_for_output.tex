% -------------------------------------------------
% Set up a new layer for the debugging marks, and make sure it is on
% top
% this is a good example: https://tex.stackexchange.com/questions/254136/how-do-i-fix-spacing-on-paths-for-a-nested-tikz-flowchart

\pgfdeclarelayer{marx}
\pgfsetlayers{main,marx}
% A macro for marking coordinates (specific to the coordinate naming
% scheme used here). Swap the following 2 definitions to deactivate
% marks.
\providecommand{\cmark}[2][]{%
  \begin{pgfonlayer}{marx}
    \node [nmark] at (c#2#1) {#2};
  \end{pgfonlayer}{marx}
  } 
\providecommand{\cmark}[2][]{\relax} 
% -------------------------------------------------

\begin{tikzpicture}[%
    >=triangle 60,              % Nice arrows; your taste may be different
    start chain=going below,    % General flow is top-to-bottom
    node distance=7mm and 55mm, % Global setup of box spacing
    every join/.style={norm},   % Default linetype for connecting boxes
    ]

{\small\ttfamily\selectfont
% ------------------------------------------------------------------------------ 
% A few box styles 
% <on chain> *and* <on grid> reduce the need for manual relative
% positioning of nodes
\tikzset{
  % ----------------------------------------------------------------------------
  % Connector line styles for different parts of the diagram
  base/.style={draw, on chain, on grid, align=center, minimum height=1cm, font={\small}},
  notes/.style={node distance=13cm, align=right},
  diam/.style={base, diamond, aspect=2, text width=5cm},
  diam_small/.style={base, diamond, aspect=2, text width=4cm},
  proc/.style={base, rectangle, text width=7cm},
  proc_small/.style={base, rectangle, text width=8em},
  term/.style={proc, rounded corners, text width=8cm},
  % coord node style is used for placing corners of connecting lines
  coord/.style={coordinate, on chain, on grid, node distance=55mm and 75mm},
  % nmark node style is used for coordinate debugging marks
  nmark/.style={draw, cyan, circle, font={\sffamily\bfseries}},
  % ----------------------------------------------------------------------------
  % Connector line styles for different parts of the diagram
  norm/.style={->, draw, blue},
  free/.style={->, draw, green},
  cong/.style={->, draw, red},
  it/.style={font={\small\itshape}}
  % ----------------------------------------------------------------------------
}
% -------------------------------------------------
% Start by placing the nodes
% Use join to connect a node to the previous one a

\node [term] (p0) {
    rule $r$;\\
    $out$, ;\\
    $minC$, threshold in the confidence\\
};
\node [diam, join, fill=blue] (is_closed) {
    $\lnot isClosed(r)$ or $conf_{pca}(r) < minC$
};
\node [proc, join, fill=red] (parents) {
    $parents$ = $parentsOfRule(r, out)$
};
\node [diam, join, fill=red] (is_empty) {
    $\lnot$ parents.isEmpty()
};
\node [proc_small, join, fill=red] (parents_dequeue) {
    $r_p$ = parents.dequeue()
};
\node [diam_small, join, fill=red] (check_confidence) {
    $conf_{pca}(r) \leq conf_{pca}(r_{p})$
};

\node [proc_small] (end) {end};
\node [proc_small, left=of end] (false) { return $false$ };
\node [proc_small, right=of end] (true) { return $true$ };

% marks
\node[coord, left=of is_closed] (c0) {}; \cmark{0} 
\node[coord, right=of is_empty] (c1) {}; \cmark{1} 
\node[coord, left=of check_confidence] (c2) {}; \cmark{2} 
\node[coord, right=of check_confidence] (c3) {}; \cmark{3} 

% labels
\path (is_closed.south) to node [midway, xshift=1em] {no} (parents); 
\path (is_closed.west) to node [midway, yshift=1em] {yes} (c0); 
\draw [*->] (is_closed.west) -- (c0) -- (c2) |- (false);

\path (is_empty.south) to node [midway, xshift=2.4em] {not empty} (parents_dequeue); 
\path (is_empty.east) to node [midway, yshift=1em] {empty} (c1); 
\draw [o->] (is_empty.east) -- (c1) -- (c3) |- (true);

\path (check_confidence.west) to node [midway, yshift=1em] {yes} (c2); 
\draw [*->] (check_confidence.west) -- (c2);

\path (check_confidence.east) to node [midway, yshift=1em] {no} (c3); 
\draw [o->] (check_confidence.east) -- (c3);

\draw[->] (false.east)  -- (end.west);
\draw[->] (true.west)  -- (end.east);

}
\end{tikzpicture}
