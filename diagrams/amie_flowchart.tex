% -------------------------------------------------
% Set up a new layer for the debugging marks, and make sure it is on
% top
% this is a good example: https://tex.stackexchange.com/questions/254136/how-do-i-fix-spacing-on-paths-for-a-nested-tikz-flowchart

\pgfdeclarelayer{marx}
\pgfsetlayers{main,marx}


\begin{tikzpicture}[%
    >=triangle 60,              % Nice arrows; your taste may be different
    start chain=going below,    % General flow is top-to-bottom
    node distance=4mm and 10mm, % Global setup of box spacing
    every join/.style={norm},   % Default linetype for connecting boxes
    ]

{\small\ttfamily\selectfont
% -------------------------------------------------
% Start by placing the nodes
% Use join to connect a node to the previous one a

\node [term] () {
    KB $\mathcal{K}$, knowledge base;\\
    $minHC$, minimum head coverage;\\
    $maxLen$, size of rules;\\
    $minConf$, threshold in the confidence\\
};
\node [proc, join] () {
    out = $\langle\rangle$ \\
    q = [$r_1(x, y)$, $r_2(x, y)$,\ldots, $r_m(x,y)$]
};
\node [diam, join] (is_empty) {
    $\lnot q.isEmpty()$
};
\node [proc_small, join] () {
    r = q.dequeue()
};
\node [diam_small, font={\small}, join] (accepted_for_output) {
    AcceptedForOutput(r, out, minConf)
};
\node [proc_small, join] (p6) {
    out.add(r)
};
\node [diam_small, join] (check_max_len) {
    len(r) < maxLen
};
\node [proc_small, join] (refinement) {
    R(r) = Refine(r)
};
\node [diam_small, join] (R_is_empty) {
    $r_c \in R(r)$
};
\node [diam_small, join] (check) {
    hc($r_c$) $\geq$ minHC \& $r_c$ $\notin$ q
};
\node [proc_small, join] (p11) {
    q.enqueue($r_c$)
};
\node [proc_small, below of = p11, node distance = 2.6cm] (return) {return out};
\node [proc_small, join] (end) {end};

% marks
\node[coord_inner_inner, right= of check] (c0) {}; \cmark{0}
\node[coord_inner_inner, below=1cm of p11] (c1) {}; \cmark{1}
\node[coord_inner_inner, left=of p11] (c2) {}; \cmark{2}
\node[coord_inner_inner, left=of R_is_empty] (c3) {}; \cmark{3}

\node[coord_inner, right=of check_max_len] (c4) {}; \cmark{4}
\node[coord_inner, below=1cm of c1] (c5) {}; \cmark{5}
\node[coord_inner, left=of is_empty] (c6) {}; \cmark{6}

\node[coord, right=of is_empty] (c7) {}; \cmark{7}

\node[coord_inner, right=of accepted_for_output] (c8) {}; \cmark{8}

%\node[coord, right=of is_empty] (c0) {}; \cmark{0}
%\node[coord_inner, right=of accepted_for_output] (c1) {}; \cmark{1}

\draw[->] (check.east) -- (c0) |- (c1) -| (c3) -- (R_is_empty);
\draw[dashed, -o] (p11.west) -- (c2);

\draw[->] (check_max_len.east) -- (c4) |- (c5) -| (c6) -- (is_empty);

\draw[->] (is_empty.east) -- (c7) |- (return);

\draw[->] (accepted_for_output.east) -- (c8) |- (check_max_len.north);

%\draw[->] (check.east) -- ++(0.66,0) -- ++(0,-0.05) -- node [near start] {no} ++(0,-3.0) -- ++(-7,0) -- ++(0, 7) -- (p9.north);
%\draw[->] (is_empty.east)  -- ++(0,0.0) -- ++(2.35,0) -- node [near start] {no} ++(0,-22.3) -- ++(-4,0) -- (return.north);
%\draw[->] (accepted_for_output.east)  -- ++(0,0.0) -- ++(1.35,0) -- node [near start] {no} ++(0,-3.3) -- (check_max_len.north);
%\draw[->] (check_max_len.east)  -- ++(0,0.0) -- ++(1.35,0) -- node [near start] {no} ++(0, -12.5) -- ++(-12,0) -- ++(0, 21) -- (is_empty.north);

% notes
%\node [notes, right of=q] {q is a queue of rules containing\\all head atoms (size 1)};
%\node [notes, right of=is_empty] {Iterate until the queue is empty};
%\node [notes, right of=refinement] {The refinement is the expansion of the rule,\\to produce a set of new rules};
%\node [notes, right of=dequeue] {Dequeue the rules};
%\node [notes, right of=accepted_for_output] {Decide whether to output a rule,\\check for closeness and connectedness,\\among other things, see algorithm 2};
%\node [notes, right of=check] {Check if the rules are pruned (by the head coverage) and duplicated};
%\node [notes, right of=check_max_len] {Check if the rule exceeds the maximum number of atoms};

}
\end{tikzpicture}
