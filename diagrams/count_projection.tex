% -------------------------------------------------
% Set up a new layer for the debugging marks, and make sure it is on
% top
% this is a good example: https://tex.stackexchange.com/questions/254136/how-do-i-fix-spacing-on-paths-for-a-nested-tikz-flowchart

\pgfdeclarelayer{marx}
\pgfsetlayers{main,marx}
% A macro for marking coordinates (specific to the coordinate naming
% scheme used here). Swap the following 2 definitions to deactivate
% marks.
\providecommand{\cmark}[2][]{%
  \begin{pgfonlayer}{marx}
    \node [nmark] at (c#2#1) {#2};
  \end{pgfonlayer}{marx}
  }
\providecommand{\cmark}[2][]{\relax}
% -------------------------------------------------

\begin{tikzpicture}[%
    >=triangle 60,              % Nice arrows; your taste may be different
    start chain=going below,    % General flow is top-to-bottom
    node distance=7mm and 55mm, % Global setup of box spacing
    every join/.style={norm},   % Default linetype for connecting boxes
    ]

{\small\ttfamily\selectfont
% ------------------------------------------------------------------------------
% Start by placing the nodes
% Use join to connect a node to the previous one a

\node [term] () {
    selection variable, $x$;\\
    a projection atom attached to a conjuctive query, $R(X, Y) \land B_1 \land B_2 \land \ldots \land B_n$;\\
    a threshold $k$;\\
    knowledge base, \KB;\\
};
\node [term, join] () {
    $map = \{\}$\\
    $q = B_1 \land B_2 \land \ldots \land B_n$
};
\node [diam, join] (x_in_B) {
    $x \in R(X, Y)$
};

\node[coord, left=of x_in_B] (c0) {}; \cmark{0}
\node[coord, right=of x_in_B] (c1) {}; \cmark{1}

%%%%%%%%%%%%%%%%%%%%%%%%%%%%%%%%%%%%%%%%%%%%%%%%%%%%%%%%%%%%%%%%%%%%%%%%%%%%%%%%
% JA
%%%%%%%%%%%%%%%%%%%%%%%%%%%%%%%%%%%%%%%%%%%%%%%%%%%%%%%%%%%%%%%%%%%%%%%%%%%%%%%%

\node [diam_small, below=of c0, fill=red] (yes_) {
    instantiations\\
    $r(x, y) \in R(X, Y)$
};
\node [proc, join, fill=red, label={a}] () {
    $q' = q$\\
    replace R by r, X by x, Y by y\\
};
\node [diam_small, join, fill=red] (exists) {
    \textsc{exists}(q', $\K$)
};
\node [proc_small, join, fill=red] (yes_result) {
    $map[x]++$
};

%%%%%%%%%%%%%%%%%%%%%%%%%%%%%%%%%%%%%%%%%%%%%%%%%%%%%%%%%%%%%%%%%%%%%%%%%%%%%%%%
% NEIN
%%%%%%%%%%%%%%%%%%%%%%%%%%%%%%%%%%%%%%%%%%%%%%%%%%%%%%%%%%%%%%%%%%%%%%%%%%%%%%%%

\node [diam_small, below=of c1, fill=green] (no_) {
    instantiations\\
    $r(x, y) \in R(X, Y)$
};
\node [proc, join, fill=green] () {
    $q' = q$\\
    replace R by r, X by x, Y by y\\
};
\node [proc, join, fill=green] () {
    $\mathcal{X} = \textsc{select}(x, q', \K)$
};
\node [diam_small, join, fill=green] (x_in_X) {
    $x \in \mathcal{X}$
};
\node [proc_small, join, fill=green] (no_result) {
    $map[x]++$
};

% marks
\node[coord_inner, left=of yes_result] (c3) {}; \cmark{3}
\node[coord_inner, below right=of exists] (c5) {}; \cmark{5}

\node[coord, below left=of x_in_X] (c6) {}; \cmark{6}

\node[coord_inner, right=of no_] (c8) {}; \cmark{8}

\node[coord_inner_inner, right=of x_in_X] (c2) {}; \cmark{2}
\node[coord_inner_inner, below=1cm of no_result] (c4) {}; \cmark{4}
\node[coord_inner, left=of no_result] (c9) {}; \cmark{9}

% ending nodes
\node [proc, below=1.5cm of c6] (return) {$map = \{ x \rightarrow n\} \in map: n \geq k$};
\node [proc, join] () {return $map$};
\node [proc, join] () {end};

% labels
\path (x_in_B.west) to node [midway, yshift=1em] {yes} (c0);
\draw [*->] (x_in_B.west) -- (c0) -- (yes_.north);

\path (x_in_B.east) to node [midway, yshift=1em] {no} (c1);
\draw [o->] (x_in_B.east) -- (c1) -- (no_.north);

\draw [dashed, -o] (yes_result.west) -- (c3);
\draw [->] (c3) |- (yes_.west);
\draw [-] (exists.east) -| (c5) -| (c3);

\draw [->] (yes_.east) -| (return.north);
\draw [-o] (no_.east) -| (c8) |- (c6);

\draw [dashed, -o] (no_result.west) -- (c9);
\draw [->] (x_in_X.east) -- (c2) |- (c4) -| (c9) |- (no_.west);

}
\end{tikzpicture}
