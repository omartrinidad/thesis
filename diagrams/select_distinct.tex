% -------------------------------------------------
% Set up a new layer for the debugging marks, and make sure it is on
% top
% this is a good example: https://tex.stackexchange.com/questions/254136/how-do-i-fix-spacing-on-paths-for-a-nested-tikz-flowchart

\pgfdeclarelayer{marx}
\pgfsetlayers{main,marx}
% A macro for marking coordinates (specific to the coordinate naming
% scheme used here). Swap the following 2 definitions to deactivate
% marks.
\providecommand{\cmark}[2][]{%
  \begin{pgfonlayer}{marx}
    \node [nmark] at (c#2#1) {#2};
  \end{pgfonlayer}{marx}
  } 
\providecommand{\cmark}[2][]{\relax} 
% -------------------------------------------------

\begin{tikzpicture}[%
    >=triangle 60,              % Nice arrows; your taste may be different
    start chain=going below,    % General flow is top-to-bottom
    node distance=7mm and 55mm, % Global setup of box spacing
    every join/.style={norm},   % Default linetype for connecting boxes
    ]

{\small\ttfamily\selectfont
% ------------------------------------------------------------------------------ 
% Start by placing the nodes
% Use join to connect a node to the previous one a

\node [term] () {
    projection variable, $x$;\\
    conjuctive query, $B_1 \land B_2 \land \ldots \land B_n$;\\
    \KB, knowledge base;\\
};
\node [term, join] () {
    $q = B_1 \land B_2 \land \ldots \land B_n$
};
\node [term, join] () {
    $s = \argmin{i}{size(B_i, \K)} $
};
\node [term, join] () {
    $result = \{\}$
};
\node [diam, join] (x_in_B) {
    $x \in B_s$
};

\node[coord, left=of x_in_B] (c0) {}; \cmark{0} 
\node[coord, right=of x_in_B] (c1) {}; \cmark{1} 

%%%%%%%%%%%%%%%%%%%%%%%%%%%%%%%%%%%%%%%%%%%%%%%%%%%%%%%%%%%%%%%%%%%%%%%%%%%%%%%%
% JA
%%%%%%%%%%%%%%%%%%%%%%%%%%%%%%%%%%%%%%%%%%%%%%%%%%%%%%%%%%%%%%%%%%%%%%%%%%%%%%%%

\node [diam_small, below=of c0, fill=red] (yes_) {
    instantiations\\$x \in x$
};
\node [proc, join, fill=red, label={a}] () {
    $q' = q$ instantianted with bindings from $b_s$
};
\node [diam_small, join, fill=red] (exists) {
    \textsc{exists}(q', $\K$)
};
\node [proc_small, join, fill=red] (yes_result) {
    $result.add(x)$
};

%%%%%%%%%%%%%%%%%%%%%%%%%%%%%%%%%%%%%%%%%%%%%%%%%%%%%%%%%%%%%%%%%%%%%%%%%%%%%%%%
% NEIN
%%%%%%%%%%%%%%%%%%%%%%%%%%%%%%%%%%%%%%%%%%%%%%%%%%%%%%%%%%%%%%%%%%%%%%%%%%%%%%%%

\node [proc_small, below=of c1, fill=green] (no_) {
    $q = q \setminus \{ B_s \}$
};
\node [diam_small, join, fill=green] (b_in_B) {
    instantiations\\$b_s \in B_s$
};
\node [proc, join, fill=green] () {
    $q' = q$ instantianted with bindings from $b_s$
};
\node [proc, join, fill=green] (no_result) {
    $result.add(\textsc{select}(x, q', \K))$
};


% marks
\node[coord_inner, left=of yes_result] (c3) {}; \cmark{3} 
\node[coord_inner, below right=of exists] (c5) {}; \cmark{5} 

\node[coord, right=of yes_result] (c6) {}; \cmark{6} 
\node[coord_inner, left=of b_in_B] (c7) {}; \cmark{7} 

\node[coord_inner, right=of b_in_B] (c8) {}; \cmark{8} 

% ending nodes
% fail
%\node [proc, below=of yes_result] at ($(no_result)!0.5!(yes_result)$) {b};

% fail
%\path (yes_result) -- (no_result) node[midway below] (b) {b};

% fail
%\node [below right=of yes_result] (a) {};
%\node [below left=of no_result] (b) {};
%\node [proc, below=of yes_result] at ($(a)!0.5!(b)$) {x};
\node [proc, below=1.5cm of c6] (return) {return $result$};
\node [proc, join] () { end };

% labels
\path (x_in_B.west) to node [midway, yshift=1em] {yes} (c0); 
\draw [*->] (x_in_B.west) -- (c0) -- (yes_.north);

\path (x_in_B.east) to node [midway, yshift=1em] {no} (c1); 
\draw [o->] (x_in_B.east) -- (c1) -- (no_.north);

\draw [->] (yes_result.west) -- (c3) |- (yes_.west);
\draw [-o] (exists.east) -| (c5) -| (c3);

\draw [->] (no_result.west) -| (c7) -- (b_in_B.west);

\draw [->] (yes_.east) -| (return.north);
\draw [-o] (b_in_B.east) -| (c8) |- (c6);

}
\end{tikzpicture}
